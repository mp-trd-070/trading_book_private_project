\documentclass[ngerman,12pt,a4paper]{book}
\usepackage[T1]{fontenc}
\usepackage{babel}
\begin{document}

\section{Trader Management}
\subsection{Ein schlechter Trade kommt selten allein}
Immer wenn man einen schlechten Trade hatte, kommt noch ein zweiter und ein dritter um die Ecke. Es liegt einfach daran, dass man höchstem körperlichen Stress ausgesetzt ist und nicht vernünftig handeln kann. Schwierige Situationen haben dann eine deutlich schlechtere xPNL. 

\section{Trade Analysis}
\subsection{PreTrade}
\subsubsection{xPNL}
Wenn ich diesen Trade 100 nacheinander mache, was kostet oder bringt mehr Trade im Schnitt multipliziert mit meinem jetzigen Risk? Log growth * Volumen des Trades. Wichtig ist, dass Trades die alles kosten können, langfristig eine xPNL von -$\inf$ haben. 
\subsection{Guter Trade mit schlechtem Ergebnis}
Trades die ein gutes Ergebnis haben, können trotzdem schlecht gewesen sein. Trades die in schlechtes Ergebnis haben, können trotzdem gut gewesen sein. Wer verdient hat nicht zwingend Recht, wer verliert nicht zwingend Unrecht.

\subsection{Loss on Top}
Man muss sich die Frage stellen, wie viel Loss man ursprünglich bereit war in dem Trade einzuzahlen und wie viel es dann am Ende tatsächlich geworden ist. Das was man zusätzlich zum eingeplanten Loss eingezahlt hat, ist der Loss on Top.


\section{Position Management}
\subsection{Aufbewahren lassen}
Wenn man eine Position hat und es aber noch erstmal in die falsche Richtung gehen wird, dann löst man die Position auf und geht später wieder in die Position. Bei Kerdos hatten wir oft den Fehler gemacht, das wir uns oben haben zushaken lassen und dann unten noch einen großen Batzen genommen haben. Der großen Batzen brachte das Geld, die Stücke oben waren in der Regel kostenpflichtig. Es wäre besser gewesen, die Stücke oben vom Markt aufbewahren zu lassen. 

\subsection{Overbetting}
Die meisten Händler handeln viel zu long in Relation zu dem Edge den sie mit der Position haben. 
\subsection{STOPS}
\subsubsection{Price Stop}
Man braucht einen Preis auf dem man ohne wenn und aber kappen geht. 

\subsubsection*{Event Stop}
Wenn man erkennt, dass der Markt Stärke zeigt entgegen der Position die man hat, sollte man nicht bis zum Stop warten sondern sich das Geld schnappen und sich statt dem Warten auf den Stop ein Steak leisten.

\subsubsection{Time Stop}
Man braucht eine Uhrzeit zu der der Trade spätestens funktioniert haben muss. Wenn ich eine kanadische Aktie habe, die ich um 10 kaufe, dann muss sie spätestens um 15:30 vorne liegen, sonst sehe ich keinen Sinn darin, die Position zu behalten. 

\subsubsection*{Bauchgefühl Stops}
Wenn sich etwas an einer Position falsch anfühlt, muss man sofort raus aus der Position. Man spürt es, wenn etwas nicht stimmt. 

I closed out all my positions on Oct.19 and 20 because I felt there was something happening in the world that I didn't understand. - Bruce Kovner in Market Wizards

\subsubsection{Der Hoffnungs-Stop}
Sobald man unterbewusst anfängt zu hoffen, dass ein Trade noch dreht muss man sofort aus der Position raus. Es ist das Zeichen des Unterbewusstseins, dass man nichts mehr in der Position verloren hat. 

\subsubsection{Let me out Event}
Besonders wenn man große Positionen hat, darf man nicht auf eine der Stop oder Take-Profit Conditions warten. Wenn man günstig aus einer Position rausgelassen wird, muss man das nutzen. 

\subsection{Nachkaufen}
\subsubsection{Der Hang zum Heldentrade}
Wenn man nachkauft hat man das Problem, das man meistens schon hinten liegt. Wer hinten liegt, agiert aus der Verteidigung heraus und dadurch im Durchschnitt schlechter als jemand der vorne liegt.
 
\subsubsection{Schnittkurstrading}
Wenn man den Schnittkurs senken will, dann nicht bloß aus dem Grund. Man muss das Gefühl haben, dass sich das zusätzliche Risk lohnt. Ich habe viele Trades von 52 gesehen, wo es immer nur darum ging, den Schnittkurs zu senken doch am Ende führt der gesenkte Schnittkurs bloß zu mehr Risk.

\subsubsection*{Zoellnern}
Wenn man kurz vom Auskotzen der Position ist, kauft man mit engem Stop nochmal die gleiche Size nach. Wichtig ist der enge Stop. 

\subsection{Gewinner laufen lassen}
Geld wird vor allem dann verdient, wenn man Gewinner möglichst lange laufen lässt. In Cosmos Holdings war ich damals dazu verdonnert die Position so lange laufen zu lassen, bis ein Stornorisiko ausgeschlossen war. Ich hatte bei 40 Cent gekauft und bei ungefähr 19 Dollar gegeben, statt wie sonst vermutlich bei 80 Cent. 

\section{Marktmikrostrukturen}
\subsection{Risk-Spillover-Events}
Jeder Marktteilnehmer kann nur eine bestimmte Menge Risk halten. Das bedeutet, dass die Arbitrageure, die den Markt stabil halten, eine ganze Weile Risk aufs Buch nehmen, bis das Maximum erreicht ist. Dann gibt es denn Spillover und sie kotzen das Risk wieder aus an die zweit-risikofreudigste Gruppe. 

\subsection{Moves die keiner versteht}
Immer wieder gibt es am Markt Moves die keiner versteht. Weil keiner sie versteht geht oft damit einher, dass keiner damit gerechnet hat. Wenn keiner damit gerechnet hat, ist in der Regel auch keiner auf den Move vorbereitet, was dazu führt, dass viele schnell umpositionieren müssen, weshalb diese Moves sehr gutes CRV bieten können.

\subsection{Crowded Trades}
Wenn ein Trade zu offensichtlich ist, machen alle den Trade. Wenn alle mit gleichen Stopparametern die gleiche Position halten, kommt es dazu, dass alle auch an den gleichen Punkten Profite mitnehmen oder kappen wollen. Daher wird der Trade vom Chance-Risiko-Verhältnis sehr unattraktiv. 


\subsection{Naivitätsarbitrage}
Wenn eine Aktie ihr Hoch bei 100 hat und bei 100 alle kaufen wollen, dann wird die Aktie künstlich zur 100 geschoben, dann kaufen sich alle Trendfollower ein und sobald die Käufer durch sind, hören die MarketMaker auf zu kaufen oder verkaufen sogar und schieben die Aktie erstmal wieder unter 100. Da gehen dann alle Trendfollower kappen und der Weg nach oben ist frei. 

\subsection{Selloffs nach guten Meldungen}
Wenn gute Meldungen abverkauft (oder schlechte Meldungen gekauft) werden, ist das eine wichtige Erkenntnis.

\subsection{Dinge die keiner haben will}
Für jede Aktie am Markt gibt es einen Preis auf dem man bereit ist, Stücke zu nehmen. Jede Lotterie hat irgendwo ihren Erwartungswert, selbst wenn dieser Erwartungswert irgendwo bei einem Cent liegt. Selbst Varta ist mit deutlich über 1 aus dem Handel gegangen, weil es auf dem Preis Nachfrage für die Aktie gegeben hat.

\subsection{Exit Liquidity}
Heutzutage kriegt jeder sofort alle Meldungen mit, weshalb es so etwas wie Zeitvorteile eigentlich kaum noch gibt. Dennoch braucht man für jede Position jemanden, der einen aus der Position rauslässt. Wenn man nicht weiß, an wen man die Position am Ende verkaufen soll, wird es schwierig werden, mit dem Trade Geld zu verdienen. 

\subsection{Positionierungsgeschwindigkeit}
Je volatiler der Markt, desto dringender wollen die Marktteilnehmer sich positionieren, weil das Risiko steigt, bei passiver Positionierung auf dem falschen Fuß erwischt zu werden. Die Spreads werden in solchen Marktphasen zwar größer, aber trotzdem ist es deutlich sinnvoller, aggressiv zu handeln.

\subsection{Take-Profit-Penalty}
Häufig wenn Spekulanten Geld verdient haben, möchten sie bloß noch aus der Position raus und nehmen schon auf Levels, die weit von dem was möglich ist, Gewinne mit. 

\subsection{Endlich rausgelassen werden}
Wenn es einen fetten Move gegeben hat und die Marktteilnehmer drauf warten mussten, reagieren zu können, dann wird es eher eine Verzerrung zu der Seite haben, die es eiliger hat. Wenn es Insolvenzmeldungen gibt, sind die ersten Preise nach einer Aussetzung vermutlich eher zu tief. Wenn es eine gute Meldung gibt in einer Aktie, in der viele Short sind, ist der erste Preis vermutlich eher zu hoch. 

\subsection{Slippage}
Die Differenz zwischen dem Preis den man erwartet hat und dem Preis den man am Ende tatsächlich erzielt hat. Wenn zum Beispiel alle am Jahrestief bei 100 einen Stop haben, dann kann es sein, dass man am Ende bei 1 Euro Slippage rauskommt. Wenn man stattdessen einen Stop bei 100.2 setzt kommt man bei 100.16 aus. Slippage wird größer je illiquider der Markt ist. 

"On Oct. 20th we wuold normally have been out of our short Eurodollar position about 40 to 50 points highger, but the market opened 240 points higher that day. Richard Dennis in Market Wizards

\section{Mentale Zustände beim Trading}
\subsection{Wenn es schlecht läuft}


\subsubsection{Position macht Meinung}
Wenn man eine Position hat, sieht jede Meldung so aus als wäre sie top für die Position. 

\subsubsection{Sudden Loss Paralysis}
Man spürt in dem Moment, dass etwas nicht so läuft, wie es soll und dass es vermutlich sehr viel Geld kosten wird, doch das Gehirn kommt nicht schnell genug zu dem Schluss, dass man die Position platt machen sollte. Ich hatte konkret den Fall, dass ich nach Autospreader Abschüssen immer so 60 Sekunden gebraucht habe, bis die Positionen dicht waren - natürlich schon viel zu spät.

\subsubsection{Wiederaufholmodus}
Wenn man einen Verlust gemacht hat, neigt man dazu, dass man den Verlust möglichst schnell wieder gut machen will. Das führt dazu, dass man die Gelegenheiten die sich bieten deutlich aggressiver handelt als man sollte (oder Gelegenheiten erfindet, falls es gar keine gibt). Man wird in der Regel nicht gut damit fahren, zu versuchen, den ganzen Verlust mit einem wilden Schwinger wieder zurück zu verdienen. 

\subsubsection{Vor Staunen vergessen zu handeln}
In absolut irren Bewegungen vergessen die Händler die Gelegenheiten zu erkennen, weil sie viel zu beschäftigt damit sind, zuzugucken was passiert. Man muss eine Position eingehen. Damals bei Cosmos hätte ich einfach abwarten und gucken können. Stattdessen habe ich eine Position aufgesetzt um zu gucken was passiert. 

\subsection{Wenn nichts geht}
\subsubsection{Overtrading aus Langeweile}
Man nimmt auch Setups aufs Buch, die in chancenreicheren Marktphasen niemals in Erwägung gezogen worden wären. Bei Kerdos haben wir immer gesagt "Irgendwas muss ich ja handeln" - und zack waren 20k weg. 

\section{Marktteilnehmer}
\subsection{primär}
\subsubsection{Non-Speculative-Positions}
Marktteilnehmer die ein Wertpapier aus strategischen Gründen (z.B. eine Übernahme) kaufen.
\subsubsection{Arbitrageure}
Trader die dafür sorgen, das Märkte inline bleiben. Sobald die Arbitrageure auf dem Markt fehlen, wird es wirklich hässlich wie damals in Gazprom, als der ADR an drei Märkten auf drei unterschiedlichen Preisen handelte, von denen keiner inline mit der Stammaktie in Russland war. 
\subsubsection{Stupid Money}
Die Leute, die als letztes aus maximaler Fomo auf den Zug aufspringen und versuchen irgendwie dabei zu sein. Das sind die Typen, die am Ende der Hausse Droneshield bei 4 kaufen und sagen see you next year auf den Malediven. Andersrum sind es auch diejenigen, die morgens um 7:30 bei -10\% ihre QBTS rausschmeißen weil der Markt 1\% tiefer steht.

\subsection{sekundär}
\subsubsection{Experten}
Man darf immer nicht vergessen, dass jeder am Markt das macht, was er gut kann. Trader traden, Analysten analysieren und Experten liefern Expertise

\section{Economics 101}
\subsection{Künstlich niedrige Preise sorgen dafür, dass das Angebot kleiner wird.}
Wenn man in Berlin die Mieten deckelt, dann werden weniger Wohnungen vermietet, weil es lukrativer ist, die Wohnungen erst zu vermieten, wenn der Mietendeckel gekippt wird als auf alten Mietverträgen sitzen zu bleiben. 
\subsection{Langfristig printen sich Zentralbanken aus allen Problemen}
Immer wenn man glaubt die Welt ist am Ende, kommt die Zentralbank und wandelt private Schulden in öffentliche. 
\end{document}